%\documentclass[preprint,showpacs,preprintnumbers,superscriptaddress,prb,floatfix,aps]{revtex4-1}
\documentclass[twocolumn,showpacs,preprintnumbers,superscriptaddress,prb,floatfix,aps,10pt]{revtex4-1}

\usepackage{amsmath,amssymb}
\usepackage{graphicx,textcomp}
\usepackage{physics} % bra-ket
\usepackage{epstopdf}
\usepackage{subfigure}
\usepackage{xfrac} % nice frac
\usepackage{multirow} % multirow table

\usepackage[version=3]{mhchem}
\usepackage{siunitx}
\sisetup{
  separate-uncertainty,
  output-open-uncertainty=[
}

\usepackage{xcolor}
\definecolor{abm}{RGB}{190,220,170}
\usepackage{todonotes}
%
\newcommand{\abm}[1]{\textcolor{orange}{ \bf [Antonio: #1] }}
% \renewcommand{\vec}[1]{\ensuremath{\mathbf{#1}}}
\newcommand*{\ham}{\hat{H}}
\newcommand*{\class}{\mathcal{C}}
\newcommand*{\wignerD}{\mathbb{D}(R)}
\newcommand*{\wignerDl}{\mathbb{D}^{(l)}(R)}
\newcommand*{\id}{\mathcal{E}}



\begin{document}

\title{VN ARPES}

\author{A. B. Mei}
\affiliation{Department of Materials Science and the Materials Research Laboratory
University of Illinois, 104 South Goodwin, Urbana, IL 61801}

\author{A. Rockett}
\affiliation{Department of Materials Science and the Materials Research Laboratory
University of Illinois, 104 South Goodwin, Urbana, IL 61801}

\author{L. Hultman}
\affiliation{Department of Physics, Chemistry and Biology (IFM), Link\"oping University, SE-581 83, Link\"oping, Sweden.}

\author{I. Petrov}
\affiliation{Department of Materials Science and the Materials Research Laboratory
University of Illinois, 104 South Goodwin, Urbana, IL 61801}
\affiliation{Department of Physics, Chemistry and Biology (IFM), Link\"oping University, SE-581 83, Link\"oping, Sweden.}

\author{J. E. Greene}
\affiliation{Department of Materials Science and the Materials Research Laboratory
University of Illinois, 104 South Goodwin, Urbana, IL 61801}
\affiliation{Department of Physics, Chemistry and Biology (IFM), Link\"oping University, SE-581 83, Link\"oping, Sweden.}

%\email{olhel@ifm.liu.se}

\begin{abstract}
...
\end{abstract}

\maketitle

A minimal-basis orthogonal tight-binding model based upon V $d$ and N $p$ orbitals is developed to accurately reproduce the electronic properties of NaCl-structure VN. Energy-momenta dispersion relations are obtained from the secular equation:
\begin{equation}
\label{eq:secular}
\left| \ham^{\alpha\beta}_{ij}(\vec{k}) - E(\vec{k})\delta{\alpha\beta}\delta_{ij} \right| = 0
\end{equation}
in which the tight-binding hamiltonian
\begin{equation}
\label{eq:ham}
\ham^{\alpha\beta}_{ij} = \sum_j v^{\alpha\beta}_{ij} e^{-i\vec{\tau}_{ij} \cdot \vec{k} }.
\end{equation}
$v^{\alpha\beta}_{ij} = \bra{i\alpha} \mathbb{V} \ket{j\beta}$ are matrix elements between orbitals $\alpha \equiv \ket{lm}$ and $\beta \equiv \ket{l'm'}$ centered on atoms $i$ and $j$.  $l$ and $m$ are azimuthal and magnetic quantum numbers. $\vec{\tau}_{ij}$ represents the bond vector connection i to j. The summation is evaluated explicitly up to second neighbors. Closed-form expressions are presented in Appendix \ref{appendix:tb}.

Intrinsic and crystallographic symmetries relations are utilized to determine the irreducible set of independent matrix elements. Transposition of atomic indices produce the relations:
\begin{equation}
\label{eq:transposition}
v_{ij}^{\alpha\alpha} = (-1)^{l+l'}v_{ji}^{\alpha\alpha}
\end{equation}
in which the prefactor arises from orbital parity.

Commutativity of $\mathbb{V}$ with $\wignerD$, the representation of symmetry R engendered by the tight-binding basis, yields the relationships:
\begin{equation}
[\mathbb{V},\wignerD] = 0
\end{equation}
or, equivalently,
\begin{equation}
\label{eq:stab}
v_{ij}^{\alpha\beta} = \sum_{\mu\nu} v_{ij}^{\alpha\beta} \left[\wignerD\right]^{\mu\alpha} \left[\wignerD\right]^{\nu\beta}.
\end{equation}
$\left[\phantom{a}\cdot\phantom{a}\right] ^{\mu\alpha}$ selects element $\mu\alpha$ of the enclosed matrix. Symmetry representations $\wignerD$ engendered on the tight binding basis are computed as described in \ref{appendix:wigner}

\section{Acknowledgements}

\bibliography{library}

\clearpage

VN pd tb parameters

    1.2783
   -0.4859
   -3.9389
    1.0091
    1.0350
   -0.1245
   -0.2168
   -0.0825
   -0.5461
    0.0509
    0.1387
   -0.1257
    0.4381
    0.2954




\appendix


\section{tight binding model}\label{appendix:tb}



% V_{\alpha\beta}^{ij}

Using this procedure, we identify 64 of 98 matrix elements as null. Of the remaining 34, only 14 comprise an irreducible set of independent parameters. We find \emph{a posteriori} that this minimal basis efficiently reproduces the electronic dispersion determined by ARPES and ab-initio density functional theory simulations.

, i.e. that the matrix elements commute with the stabilizer group symmetries of the bonds and,

 that 

The procedure reduces the number of independent matrix to an irreducible set of 14 parameters, which we find \emph{a posteriori} efficiently reproduces the electronic dispersion as determined by ARPES and ab initio density functional theory simulations.

The electronic band dispersion $E(\vec{k})$ of B1 NaCl-structure cubic-VN is parameterized with an orthogonal tight-binding basis based on V 3d and N 2p orbitals as described in \ref{appendix:tb}. We find \emph{a posteriori} that this minimal basis accurately and efficiently reproduces the electronic dispersion determined by ARPES and ab-initio density functional theory simulations.

$E(\vec{k})$ is computed by diagonalizing the hamiltonian $\ham$ via the secular equation:
\begin{equation}
\left| \ham_{\alpha\beta} - E(\vec{k}) \delta_{\alpha\beta} \right| = 0
\end{equation}



\begin{equation}
H = H_0 + H_1 + H_2
\end{equation}
The subscripts index neighbors.




Zeroth neighbor interactions involve energy integrals between orbitals located on the same site. In the orthogonal pd basis, these interactions produce a dispersionless diagonal matrix with elements representing band energies at $\Gamma$ the Brillouin-zone center. 
% v1 = E_eg
% v2 = E_t2g
\begin{equation}
H_0 =
\begin{bmatrix}
H_0^{dd} & 0 \\
 0 & H_0^{pp} \\
\end{bmatrix}
\end{equation}

The octahedral environment at anion sites preserve the three-fold degeneracy of N $p$ orbitals, yielding one irreducible representation, $p = T_{1u}$. Matrix elements between $p$ orbitals, $p \otimes p$, transform as $A_{1g} \oplus E_g \oplus T_{1g} \oplus T_{2g}$. The \emph{single} occurrence of the identity representation $A_{1g}$ signifies the existence of \emph{one} symmetry-adapted tight-binding matrix element, corresponding to the $p$-orbital energy $\epsilon_p$.

\begin{equation}
H_0^{pp} =
\begin{bmatrix}
\epsilon_{p} & 0 & 0 \\
0 & \epsilon_{p} & 0 \\
0 & 0 & \epsilon_{p} \\
\end{bmatrix}
\end{equation}

Cation-site symmetries lift the five-fold degeneracy on V $d$ orbitals engendering the two and three-dimension irreducible representations $E_g$ and $T_{2g}$. The outer product $d \otimes d$ decomposes into $2 A_{1g} \oplus A_{2g} \oplus 2 E_g \oplus 3 T_{1g} \oplus 3 T_{2g}$. Here, $A_{1g}$ occurs twice, indicating the existence of two independent matrix elements, which correspond to the energies $\epsilon_e$ and $\epsilon_t$ of V $d$ bands associated with $E_g$ and $T_{2g}$.

\begin{equation}
H_0^{dd} =
\begin{bmatrix}
 \epsilon_{e} & 0 & 0 & 0 & 0 \\
 0 & \epsilon_{e} & 0 & 0 & 0 \\
 0 & 0 & \epsilon_{t} & 0 & 0 \\
 0 & 0 & 0 & \epsilon_{t} & 0 \\
 0 & 0 & 0 & 0 & \epsilon_{t} \\
\end{bmatrix}
\end{equation}

The NaCl lattice consists of two interpenetrating cation and anion FCC sublattices translated half a unit-cell's length with respect to each other. Each cation (anion) has six nearest-neighbor anion (cations) at $\left[0 0 \sfrac{1}{2}\right]$. The bond stabilizer, i.e. the group of symmetries which leave the bond invariant is $C_{4v}$. $C_{4v}$ consists of eight elements: the identity $\id$, four (two $c_{4}^2$ and two $c_4$) four-fold rotations about the bond axis and four reflection planes, which contain the bond. The vertical mirrors $\sigma_v$ are orthogonal to each other and span the $xz$ and $yz$ planes. The diagonal mirrors $\sigma_d$ bisect the angles form between the $xz$ and $yz$ planes and are, thus, oriented by $\pi/2$ with respect to $\sigma_v$.

\begin{equation}
H_1 = 2i
\begin{bmatrix}
0                  & -H_1^{pd} \\
{H_1^{pd}}^\dagger &  0        \\
\end{bmatrix}
\end{equation}

in which 

\begin{equation}
H_1^{pd} = 
\begin{bmatrix}
-\sqrt{3} s_1 v_{ep} & 0            & \sqrt{3} s_3 v_{ep}  \\ % v5 = V(eg,p)
 s_1 v_{ep}          &-2 s_2 v_{ep} &  s_3 v_{ep}          \\ % v4 = V(t2g,p)
 s_2 v_{tp}          &  s_1 v_{tp}  & 0                    \\
0                    &  s_3 v_{tp}  &  s_2 v_{tp}          \\
 s_3 v_{tp}          & 0            &  s_1 v_{tp}          \\
\end{bmatrix}
\end{equation}

Second-neighbor interactions between the twelve equivalent species located at $\left[0\sfrac{1}{2}\sfrac{1}{2}\right]$. The stabilizer group is the Abelian group $C_{2v}$. The $O_h$ group has twelve $C_{2v}$ subgroups, of which six are maximal subgroups (the remaining six belong to the intermediate group $C_{4v}$). \cite{wadhawan_introduction_2000} The six maximal $C_2v$ subgroups stabilize the twelve second-neighbor bonds (two collinear bonds per subgroup). 

The group contains three elements, in addition to $\id$, corresponding to $c_2$ two-fold rotations about the bond axis $\left[0\sfrac{1}{2}\sfrac{1}{2}\right]$ and two orthogonal mirror planes $\sigma_v$ and $\sigma_v'$ containing the bond.


\begin{equation}
H_2 = 4
\begin{bmatrix}
H_2^{dd} & 0 \\
 0 & H_2^{pp} \\
\end{bmatrix}
\end{equation}

in which
\begin{widetext}
\begin{equation}
H_2^{dd} =
\begin{bmatrix}
        (c_{x} (c_{z}-2 c_{y})-2 c_{yz}) v_{6} /\sqrt{3}
                     +(c_{yz}+c_{x} (c_{y}+c_{z})) v_{7} &            c_{y} (c_{z}-c_{x}) v_{6}                         &                  -\sqrt{3} s_{xy} v_{8}  &     \sqrt{3} s_{yz} v_{8}                 &                                        0  \\
                               c_{y} (c_{z}-c_{x}) v_{6} &   (c_{yz}+c_{x} (c_{y}+c_{z})) v_{7}  -\sqrt{3} c_{zx} v_{6} &                           -s_{xy} v_{8}  &             -s_{yz} v_{8}                 &                           2 s_{zx} v_{8}  \\
                                  -\sqrt{3} s_{xy} v_{8} &                        -s_{xy} v_{8}                         & (c_{x}+c_{y}) c_{z} v_{11}+c_{xy} v_{9}  &             -s_{zx} v_{10}                &                             s_{yz} v_{10} \\
                                   \sqrt{3} s_{yz} v_{8} &                        -s_{yz} v_{8}                         &                           -s_{zx} v_{10} & c_{x} (c_{y}+c_{z}) v_{11} +c_{yz} v_{9}  &                            -s_{xy} v_{10} \\
                                                       0 &                       2 s_{zx} v_{8}                         &                           -s_{yz} v_{10} &             -s_{xy} v_{10}                & c_{y} (c_{x}+c_{z}) v_{11} +c_{zx} v_{9}  \\
\end{bmatrix}                                                                                                                                                                        
\end{equation}



\begin{equation}
H^{pp}_2 =
\begin{bmatrix}
               c_{yz} v_{12} +c_{x} (c_{y}+c_{z}) v_{14} &              -s_{xy} v_{13}                             &              -s_{zx} v_{13}                            \\
              -s_{xy} v_{13}                             &               c_{zx} v_{12} +c_{y} (c_{z}+c_{x}) v_{14} &              -s_{yz} v_{13}                            \\
              -s_{zx} v_{13}                             &              -s_{yz} v_{13}                             &               c_{xy} v_{12}+ c_{z} (c_{x}+c_{y}) v_{14} \\
\end{bmatrix}
\end{equation}
\end{widetext}

In Eqs. XX - XX,
\begin{align}
\begin{split}
c_x &= \cos(\pi a k_x) \\
c_y &= \cos(\pi a k_y) \\
c_z &= \cos(\pi a k_z)
\end{split}
\begin{split}
s_x &= \sin(\pi a k_x) \\
s_y &= \sin(\pi a k_y) \\
s_z &= \sin(\pi a k_z)
\end{split}
\end{align}
and
$s_{ij} = s_i s_j$ and $c_{ij} = c_i c_j$.




%\section{Film growth and characterization}

%
% WIGNER D MATRIX
%
\section{wigner D matrix} \label{appendix:wigner}

The symmetry representations $\wignerD$ engendered by the $p \oplus d$ tight-binding basis are:
\begin{equation}
D(R) = \bigoplus_l D^{(l)}(R)
\end{equation}
wherein the Wigner matrices $\wignerDl =  e^{-i\theta\hat{n}\cdot\hat{L}}$ parameterize rotations $\theta$ around $\hat{n}$ and $\hat{L}$ is the angular momentum operator. % Lax p 43 Eq 2.5.18





Wigner matrices $D^{(l)}(R)$ corresponding to $(2l+1)$-dimensional irreducible square matrix representations of the three-dimensional special orthogonal group SO(3) are obtained from:\cite{martin_electronic_2004}
\begin{equation}
\label{eq:imselfenergy}
% Shankar p 333, Exercise 12.5.7
% Wolfram p 85-87
% In the book, Euler angles are used. Here, I use Tait Bryan angles
D^{(l)}(R) = \bra{l',m'} R \ket{l,m}
\end{equation}
with rotations about axes $\hat{x}$, $\hat{y}$, and $\hat{z}$ parameterized by Tait Bryan angles $\theta$, $\phi$, and $\psi$:
\begin{equation}
R = e^{-i\theta\hat{L}_x/\hbar} e^{-i\phi\hat{L}_y/\hbar} e^{-i\psi\hat{L}_z/\hbar}
\end{equation}
%
In the $\hat{L}_y$ basis, the angular momenta matrix elements are:\cite{shankar_fundamentals_2014}
\begin{align}
\label{eq:angular_momenta}
% x y z <- z x y
% Shankar p 327-328, Eq. 12.5.21a-b
% Wolfram p 85-87
\hat{L}_x & = \frac{\hat{L}_{+}-\hat{L}_{-}}{2i} \hbar \\
\hat{L}_y & = \delta_{l,l'}\delta_{m,m'} \hbar m \\
\hat{L}_z & = \frac{\hat{L}_{+}+\hat{L}_{-}}{2} \hbar
\end{align}
with raising $\hat{L}_+$ and lowering $\hat{L}_-$ matrix elements:
\begin{equation}
\label{eq:raising_lowering_operator}
% Shankar p 327, Eq. 12.5.20
\hat{L}_{\pm} = \delta_{l,l'}\delta_{m\pm1,m'} \sqrt{(j\mp m)(j\pm m+1)} ,
\end{equation}

With the scalar $(-1)^l$ representing inversion, the full three-dimensional orthogonal group $O(3)$, which includes improper rotations, may be produced from the products:

Irreducible matrix representations corresponding to inversion are obtained as diagonal matrices with entries equal to $(-1)^l$. The product $(-1)^{l} D^{(l)}(R)$  can thus produce any irreducible representations of the full three-dimensional orthogonal group O(3).\cite{sharma_general_1979,el-batanouny_symmetry_2008} % Wooten p 604 bottom, under Eq. 14.164 




%
% IRREDUCIBLE CHARACTERS
%
\section{Calculation of irreducible characters}

The $O_h$ point group consists of forty-eight symmetry operations which restore the NaCl-structure lattice onto itself. Twenty-four symmetries are proper rotations belonging to the chiral octahedral subgroup $O$. These are subdivided among five conjugacy classes\footnote{Classes $\class_i$ are mutually distinct sets containing symmetries $R_i$ and $R_j$ related by conjugation with $R_k$ an arbitrary group element: $R_kR_iR_k^{-1}=R_j$} --  $\id$, $3c_4^2$, $6c_2$, $8c_3$, and $6c_4$ -- which correspond to the identity, two-fold rotations about $\left<001\right>$, two-fold rotations around $\left<011\right>$, three-fold rotations about $\left<111\right>$, and four-fold rotations around $\left<001\right>$. The additional twenty-four symmetries, formed by compounding $O$ symmetries with the inversion operator $i$, are improper rotations\footnote{Improper rotations, or rotoinversions, are rotation operations followed by inversion.} that exhibit a similar structure of five conjugacy classes: $i$, $3\sigma_2$, $6\sigma_2$, $8\sigma_6$, and $6\sigma_4$, associated with inversion $i$, reflection across $\{001\}$, reflection across $\{011\}$, three-fold roto-inversions around $\left<111\right>$, and four-fold roto-inversions about $\left<001\right>$. 

[should already have introduced representations and irreps]

$O_h$ exhibits ten irreducible representations $\Gamma^{(\mu)}$. These are labeled according to both Mulliken\footnote{In Mulliken notation, letters A,E,G, and H denote irreducible representations of dimensions one, two, three, and four. B is used in lieu of A for symmetries which are antisymmetric with respect to the principal axis (the axis of highest cyclical symmetry). Subscripts pairs $g$ and $u$, $1$ and $2$ as well as single and double primes are used to denote representations displaying even and odd parities with respect to inversion, horizontal two-fold rotations $c_2$, and horizontal reflections $\sigma_h$.} and BSW.


The group is non-Abelian with . The two group generators, $\sigma_6$ and $c_4$, fully define symmetry relations.
matrix element 


The three dimensional representations arise from non-commuting three-fold and six-fold symmetry elements. 




As required by translational symmetry, the crystallographic point groups exhibit integer characters.





$O_h$



All ten $O_h$ conjugacy classes produced by the direct product $O \otimes C_i$ are presented in the first row of Table \ref{table:chi}.




A summary of irreducible representation characters for the $O_h$ point group are presented in 
Table \ref{table:chi}. Also shown are the characters for irreducible representations of ... 

Characters of irreducible representations are determined from conjugacy class coefficients\cite{burnside_theory_2010,mckay_construction_1970,unger_computing_2006,schneider_dixons_1990,dixon_high_1967}.

Symmetry groups decompose into disjoint classes $\class_i$ consisting of distinct and mutually conjugate symmetry elements. Two elements A and B belong to the same class if there exists a symmetry element $X$ in the group, for which $XA = BX$ is satisfied.

The number of times a class $\class_k$ appears in the decomposition of products of classes $\class_i$ and $\class_j$ is $h_{jk}^{(i)}$. 

\begin{equation}
\label{eq:class_coefficients}
\class_i \class_j = \sum_k h_{jk}^{(i)} \class_k
\end{equation}
with expansion coefficients $h_{ijk}$.

% simultaneous dec
The similarity transform $\Lambda$:
\begin{equation}
\Lambda^{-1} P \Lambda = \lambda
\end{equation}
which simultaneously diagonalizes the $i$ commuting expansion matrices $h_{jk}$ is determined using the matrix pencil:
\begin{equation}
\label{eq:matrix_pencil}
P_{jk} = \sum_i r_i h_{ijk}
\end{equation}
with arbitrary factors $r_i$ added to lift eigendegeneracies.


Characters corresponding to irreducible representation $\alpha$ and conjugacy class i are thus obtained as:
\begin{equation}
\label{eq:irrep_characters}
\chi\left(\Gamma_i^\alpha\right) = \frac{d_\alpha}{n_i} \lambda_{i\alpha}
\end{equation}
in which number of elements in the class $i$ is $n_i$ and the dimensions of irreducible representation $\alpha$:
\begin{equation}
\label{eq:irrep_dimension}
d_\alpha = \sqrt{ \frac{g}{\sum_i \lambda_{i\alpha} \lambda_{i\alpha}^\dag / n_i }  }
\end{equation}
g is the order of the group.


%
%
%
\section{Calculation of irreducible characters}

\begin{table}
\caption{\label{table:chi} $O_h$ representation characters}
\begin{ruledtabular}
\begin{tabular*}{10cm}{llrrrrrrrrrr}
\multicolumn{2}{c}{$O_h$}         &$\id$&3$c_4^2$& 6$c_2$ & 8$c_3$ & 6$c_4$ &  $i$ & 3$\sigma_2$ & 6$\sigma_2$ & 8$\sigma_6$ & 6$\sigma_4$ \\  
$A_{1g}$        & $\Gamma_{1}  $  &  1  &     1  &     1  &     1  &     1  &   1  &          1  &          1  &          1  &          1  \\         %  irrep =  1 
$A_{2g}$        & $\Gamma_{2}  $  &  1  &     1  &    -1  &     1  &    -1  &   1  &          1  &         -1  &          1  &         -1  \\         %  irrep =  2 
$E_g   $        & $\Gamma_{12} $  &  2  &     2  &     0  &    -1  &     0  &   2  &          2  &          0  &         -1  &          0  \\         %  irrep =  3 
$T_{1g}$        & $\Gamma_{15} $  &  3  &    -1  &    -1  &     0  &     1  &   3  &         -1  &         -1  &          0  &          1  \\         %  irrep =  5 
$T_{2g}$        & $\Gamma_{25} $  &  3  &    -1  &     1  &     0  &    -1  &   3  &         -1  &          1  &          0  &         -1  \\         %  irrep =  4 
$A_{1u}$        & $\Gamma_{1} '$  &  1  &     1  &     1  &     1  &     1  &  -1  &         -1  &         -1  &         -1  &         -1  \\         %  irrep =  7 
$A_{2u}$        & $\Gamma_{2} '$  &  1  &     1  &    -1  &     1  &    -1  &  -1  &         -1  &          1  &         -1  &          1  \\         %  irrep =  6 
$E_u   $        & $\Gamma_{12}'$  &  2  &     2  &     0  &    -1  &     0  &  -2  &         -2  &          0  &          1  &          0  \\         %  irrep =  8 
$T_{1u}$        & $\Gamma_{15}'$  &  3  &    -1  &    -1  &     0  &     1  &  -3  &          1  &          1  &          0  &         -1  \\         %  irrep =  9 
$T_{2u}$        & $\Gamma_{25}'$  &  3  &    -1  &     1  &     0  &    -1  &  -3  &          1  &         -1  &          0  &          1  \\ \hline  %  irrep = 10 
\multicolumn{2}{c}{$C_{4v}$}      &$\id$& $c_4^2$&        &        & 2$c_4$ &      & 2$\sigma_v$ &  $\sigma_d$ &             &             \\ 
$A_1$           & $\Delta_{1}  $  &  1  &     1  &     .  &     .  &     1  &   .  &          1  &          1  &          .  &          .  \\         %  irrep =  1
$A_2$           & $\Delta_{1}' $  &  1  &     1  &     .  &     .  &     1  &   .  &         -1  &         -1  &          .  &          .  \\         %  irrep =  4
$B_1$           & $\Delta_{2}  $  &  1  &     1  &     .  &     .  &    -1  &   .  &          1  &         -1  &          .  &          .  \\         %  irrep =  3
$B_2$           & $\Delta_{2}' $  &  1  &     1  &     .  &     .  &    -1  &   .  &         -1  &          1  &          .  &          .  \\         %  irrep =  2
$E$             & $\Delta_{5}  $  &  2  &    -2  &     .  &     .  &     0  &   .  &          0  &          0  &          .  &          .  \\ \hline  %  irrep =  5
\multicolumn{2}{c}{$C_{2v}$}      &$\id$&        &  $c_2$ &        &        &      &  $\sigma_v$ & $\sigma_v'$ &             &             \\
$A_{1}$         & $\Sigma_{1}  $  &  1  &     .  &     1  &     .  &     .  &   .  &          1  &          1  &          .  &          .  \\
$A_{2}$         & $\Sigma_{2}  $  &  1  &     .  &     1  &     .  &     .  &   .  &         -1  &         -1  &          .  &          .  \\
$B_{1}$         & $\Sigma_{3}  $  &  1  &     .  &    -1  &     .  &     .  &   .  &         -1  &          1  &          .  &          .  \\
$B_{2}$         & $\Sigma_{4}  $  &  1  &     .  &    -1  &     .  &     .  &   .  &          1  &         -1  &          .  &          .  \\ \hline
\multicolumn{2}{c}{$C_{3v}$}      &$\id$&        &        & 2$c_3$ &        &      &             & 3$\sigma_v$ &             &             \\
$A_{1}$         & $\Lambda_{1} $  &  1  &     .  &     .  &     1  &     .  &   .  &          .  &          1  &          .  &          .  \\
$A_{2}$         & $\Lambda_{2} $  &  1  &     .  &     .  &     1  &     .  &   .  &          .  &         -1  &          .  &          .  \\
$E$             & $\Lambda_{3} $  &  2  &     .  &     .  &    -1  &     .  &   .  &          .  &          0  &          .  &          .  \\ \hline
\multicolumn{2}{c}{$p$          } &  3  &     -1 &    -1  &     0  &     1  &  -3  &          1  &          1  &          0  &         -1  \\
\multicolumn{2}{c}{$d$          } &  5  &      1 &     1  &    -1  &    -1  &   5  &          1  &          1  &         -1  &         -1  \\
\multicolumn{2}{c}{$p \oplus  d$} &  8  &      0 &     0  &    -1  &     0  &   2  &          2  &          2  &         -1  &         -2  \\
\multicolumn{2}{c}{$p \otimes p$} &  9  &      1 &     1  &     0  &     1  &   9  &          1  &          1  &          0  &          1  \\
\multicolumn{2}{c}{$p \otimes d$} & 15  &     -1 &    -1  &     0  &    -1  & -15  &          1  &          1  &          0  &          1  \\
\multicolumn{2}{c}{$d \otimes d$} & 25  &      1 &     1  &     1  &     1  &  25  &          1  &          1  &          1  &          1  \\
\end{tabular*}
\end{ruledtabular}
\end{table}


\begin{table}
\caption{\label{table:subduction} $O_h$ subductions}
\begin{ruledtabular}
\begin{tabular*}{10cm}{llll}
$O_h$    & $C_{4v}$         & $C_{2v}$                              &   $C_{3v}$            \\ \hline
$A_{1g}$ & $A_1$            & $A_{1}$                               &   $A_1$               \\ 
$A_{2g}$ & $B_1$            & $B_{1}$                               &   $A_2$               \\ 
$E_g   $ & $A_1 \oplus B_1$ & $A_{1} \oplus B_{1}$                  &   $E$                 \\ 
$T_{1g}$ & $A_2 \oplus E$   & $A_{2} \oplus B_{1} \oplus B_{2}$     &   $A_2 \oplus E$      \\ 
$T_{2g}$ & $B_2 \oplus E$   & $A_{1} \oplus A_{2} \oplus B_{2}$     &   $A_1 \oplus E$      \\ 
$A_{1u}$ & $A_2$            & $A_{2}$                               &   $A_2$               \\ 
$A_{2u}$ & $B_2$            & $B_{2}$                               &   $A_1$               \\ 
$E_u   $ & $A_2 \oplus B_2$ & $A_{2} \oplus B_{2}$                  &   $E$                 \\ 
$T_{1u}$ & $A_1 \oplus E$   & $A_{1} \oplus B_{1} \oplus B_{2}$     &   $A_1 \oplus E$      \\ 
$T_{2u}$ & $B_1 \oplus E$   & $A_{1} \oplus A_{2} \oplus B_{1}$     &   $A_2 \oplus E$      \\ 
\end{tabular*}
\end{ruledtabular}
\end{table}

Reducible representations are decomposed into $\alpha$ irreducible representations:
\begin{equation}
\label{eq:irrep_decomposition}
\Gamma = \bigoplus_\alpha n_\alpha \Gamma_{i\alpha}
\end{equation}

in which the expansion coefficients are:
\begin{equation}
\label{eq:irrep_decomposition_coefficients}
% Wolfram p 20
n_\alpha = \frac{1}{g} \sum_i h_i \chi\left(\Gamma_{i\alpha}\right)^\dag \chi\left(\Gamma_i\right)
\end{equation}
The sum is over classes $i$, with number of elements $h_i$, and characters corresponding to the $\alpha$ irreducible representation $\chi(\Gamma_i^\alpha)$. $g$ is the group order.





%
%
%
\section{Calculation of Tait Bryan angles from rotational matrices}
Tait Bryan angles $\theta$, $\phi$, and $\gamma$, corresponding to rotations around $\hat{z}$, $\hat{y}$, and $\hat{x}$ axes, are obtained from the rotational matrix:
\begin{equation}
\label{eq:rotation}
R = R_x(\theta)R_y(\phi)R_z(\gamma)
\end{equation}
via the relationships:
\begin{align}
\label{eq:tait_bryan_angles}
% This solution avoids gimbal lock cases and does not require conditional statements.
%i = 1; j = 2; k = 3
%euler(i) = atan2(R(l,k),R(k,k))
%euler(l) = atan2(-R(i,k),sqrt(R(i,i)**2+R(i,j)**2))
%s=sin(euler(1)); c=cos(euler(1))
%euler(k) = atan2(s*R(k,i)-c*R(l,i),c*R(l,j)-s*R(k,j))
\theta & = {\rm atan2}\left(R_{23},R_{33}\right) \\
\phi   & =-{\rm atan2}\left(-R_{13},\sqrt{R_{11}^2+R_{12}^2}\right) \\
\gamma & =-{\rm atan2}\left(R_{31}\sin{\theta}-R_{21}\cos{\theta},R_{22}\cos{\theta}-R_{32}\sin{\theta}\right)
\end{align}
in which ${\rm atan2}$ is the four-quadrant arctangent function and $R_{ij}$ are the elements corresponding to row i and column j of R. 


\end{document}

