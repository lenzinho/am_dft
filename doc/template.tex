%\documentclass[preprint,showpacs,preprintnumbers,superscriptaddress,prb,floatfix,aps]{revtex4-1}
\documentclass[twocolumn,showpacs,preprintnumbers,superscriptaddress,prb,floatfix,aps,10pt]{revtex4-1}

\usepackage{amsmath,amssymb}
\usepackage{graphicx,textcomp}
\usepackage{physics} % bra-ket
\usepackage{epstopdf}
\usepackage{subfigure}
\usepackage{xfrac} % nice frac
\usepackage{multirow} % multirow table

\usepackage[version=3]{mhchem}
\usepackage{siunitx}
\sisetup{
  separate-uncertainty,
  output-open-uncertainty=[
}

\usepackage{xcolor}
\definecolor{abm}{RGB}{190,220,170}
\usepackage{todonotes}
%
\newcommand{\abm}[1]{\textcolor{orange}{ \bf [Antonio: #1] }}
% \renewcommand{\vec}[1]{\ensuremath{\mathbf{#1}}}
\newcommand*{\ham}{\hat{H}}
\newcommand*{\class}{\mathcal{C}}
\newcommand*{\wignerD}{\mathbb{D}(R)}
\newcommand*{\wignerDl}{D^{(l)}(R)}
\newcommand*{\id}{\mathcal{E}}
\newcommand*{\zeromat}{0}
\newcommand*{\x}{$\times$}
\newcommand*{\bondvec}{\vec{\tau}_{ij}}
\newcommand{\seitz}[2]{\{#1|#2\}}



\begin{document}

\title{VN ARPES}

\author{A. B. Mei}
\affiliation{Department of Materials Science and the Materials Research Laboratory
University of Illinois, 104 South Goodwin, Urbana, IL 61801}

\author{A. Rockett}
\affiliation{Department of Materials Science and the Materials Research Laboratory
University of Illinois, 104 South Goodwin, Urbana, IL 61801}

\author{L. Hultman}
\affiliation{Department of Physics, Chemistry and Biology (IFM), Link\"oping University, SE-581 83, Link\"oping, Sweden.}

\author{I. Petrov}
\affiliation{Department of Materials Science and the Materials Research Laboratory
University of Illinois, 104 South Goodwin, Urbana, IL 61801}
\affiliation{Department of Physics, Chemistry and Biology (IFM), Link\"oping University, SE-581 83, Link\"oping, Sweden.}

\author{J. E. Greene}
\affiliation{Department of Materials Science and the Materials Research Laboratory
University of Illinois, 104 South Goodwin, Urbana, IL 61801}
\affiliation{Department of Physics, Chemistry and Biology (IFM), Link\"oping University, SE-581 83, Link\"oping, Sweden.}

%\email{olhel@ifm.liu.se}

\begin{abstract}

This highly-efficient basis is able to accurately reproduce both electronic band dispersion determined by ARPES and ab-initio density functional theory simulations with a minimal set of fourteen empirical fitting parameters.
\end{abstract}

\maketitle


A minimal-basis orthogonal tight-binding model based upon V $d$ and N $p$ orbitals is developed, accurately reproducing the electronic properties of NaCl-structure VN. Energy-momenta dispersion relations are obtained from the secular equation:
\begin{equation}
\label{eq:secular}
\left| \ham^{\alpha\beta}_{ij}(\vec{k}) - E(\vec{k})\delta{\alpha\beta}\delta_{ij} \right| = 0
\end{equation}
in which the tight-binding hamiltonian
\begin{equation}
\label{eq:ham}
\ham^{\alpha\beta}_{ij} = \sum_j v^{\alpha\beta}_{ij} e^{-i\vec{\tau}_{ij} \cdot \vec{k} }.
\end{equation}
$v^{\alpha\beta}_{ij} = \bra{i\alpha} \mathbb{V} \ket{j\beta}$ are matrix elements between orbitals $\alpha \equiv \ket{lm}$ and $\beta \equiv \ket{l'm'}$ centered on atoms $i$ and $j$.  $l$ and $m$ are azimuthal and magnetic quantum numbers. $\bondvec$ represents the bond connecting $i$ to $j$. The summation is evaluated explicitly to second neighbor.

Intrinsic and crystallographic symmetries relations are utilized to determine an irreducible set of matrix elements. 

Commutativity of $\mathbb{V}$ with $\wignerD$, the representation of symmetry R engendered by the tight-binding basis, yields the relationships:
\begin{equation}
[\mathbb{V},\wignerD] = 0
\end{equation}
or, equivalently,
\begin{equation}
\label{eq:stab}
v_{ij}^{\alpha\beta} = \sum_{\mu\nu} v_{ij}^{\alpha\beta} \left[\wignerD\right]^{\mu\alpha} \left[\wignerD\right]^{\nu\beta}.
\end{equation}
$\left[\phantom{.}\cdot\phantom{.}\right] ^{\mu\nu}$ selects element $\mu\nu$ of the enclosed matrix. $\wignerD$ are symmetry representations engendered on the tight binding basis and describe the coupling between states with different magnetic quantum numbers under crystallographic symmetry operations. 

Point symmetry operations in the tight binding basis are represented as direct sums over azimuthal quantum numbers of primitive atom orbitals:
\begin{equation}
\wignerD = \bigoplus_l \wignerDl.
\end{equation}
The Wigner functions $\wignerDl =  e^{-i\theta\hat{L}\cdot\hat{n}/\hbar}$ are $(2l+1)$-dimensional square matrix which parameterize proper rotations $\theta$ around axes $\hat{n}$, with $\hat{L}$ being the angular momentum operator. % Lax p 43 Eq 2.5.18
Improper rotations are obtained by premultiplying $(-1)^l$ the scalar inversion operator. \cite{sharma_general_1979,el-batanouny_symmetry_2008} % Wooten p 604 bottom, under Eq. 14.164 
Explicit expressions for $\wignerD$ are provided in Appendix \ref{appendix:wigner}




\section{Acknowledgements}

\bibliography{library}

\clearpage

VN pd tb parameters

    1.2783
   -0.4859
   -3.9389
    1.0091
    1.0350
   -0.1245
   -0.2168
   -0.0825
   -0.5461
    0.0509
    0.1387
   -0.1257
    0.4381
    0.2954




\appendix


\section{tight binding model}\label{appendix:tb}

The B1 NaCl-structure lattice consists of two interpenetrating cation and anion face-centered cubic sublattices translated by half a primitive-cell with respect to each other. Each cation (anion) has six equivalent first neighbor anions (cations) at $\vec{\tau}_{ij} = \left[0 0 \frac{1}{2}\right]$ and twelve equivalent second neighbor cation (anion) at $\vec{\tau}_{ij} = \left[0 \frac{1}{2} \frac{1}{2}\right]$. Explicitly describing interactions to second neighbors with a minimal basis based on $d$ and $p$ orbitals centered on cation and anion sites yields the tight-binging hamiltonian: 
\begin{equation}
\label{eq:ham_explicit}
\hat{H} = 
\begin{bmatrix}
\hat{H}_0^{dd} & \zeromat \\
\zeromat & \hat{H}_0^{pp} \\
\end{bmatrix}
 + 2i
\begin{bmatrix}
\zeromat              & -\hat{H}_1^{pd} \\
\hat{H}_1^{pd\dagger} &  \zeromat        \\
\end{bmatrix}
 + 4
\begin{bmatrix}
\hat{H}_2^{dd} & \zeromat       \\
\zeromat       & \hat{H}_2^{pp} \\
\end{bmatrix}.
\end{equation}
Sub- and superscripts on $\ham_{j}^{\alpha\beta}$ denote coupling of orbitals $\alpha$ and $\beta$ separated $j$ neighbor shells apart. The zero submatrices $\zeromat$ arise from the orthogonality of basis functions, which requires that matrix elements between nonequivalent sites ($i \neq j$) vanish.

The first matrix in Eq. \ref{eq:ham_explicit} is the Hamiltonian $\ham_0$ corresponding to the $0$\textsuperscript{th}-neighbor shell and therefore contains energy integrals between orbitals located on the same site (i.e., $\vec{\tau}_{ij} = 0$). Consequentially, coupling coefficients are independent of $\vec{k}$ wavevector (see Eq. \ref{eq:ham}) and $H_0$ acquires a block diagonal structure in which the first 5 \x 5 and latter 3 \x 3 entries correspond to V and N orbital self-interactions.

Octahedral $O_h$ symmetry preserves the three-fold degeneracy of N $p$ orbitals, yielding one irreducible representation, $p \rightarrow T_{1u}$. Matrix elements between $p$ orbitals transform as the outer product  $p \otimes p \rightarrow A_{1g} \oplus E_g \oplus T_{1g} \oplus T_{2g}$. The \emph{single} occurrence of $A_{1g}$ the identity representation signifies the existence of \emph{one} symmetry-adapted tight-binding matrix element. The sole independent matrix element corresponds to $\epsilon_p$ the energy of triply-degenerate $p$ orbitals. The Hamiltonian block describing zero-th neighbor overlap between N orbitals is thus:

\begin{equation}
H_0^{pp} =
\begin{bmatrix}
\epsilon_{p} & 0 & 0 \\
0 & \epsilon_{p} & 0 \\
0 & 0 & \epsilon_{p} \\
\end{bmatrix}
\end{equation}

The five-fold degeneracy on V 3$d$ orbitals are lifted by $O_h$ symmetry, producing the combination of two and three dimensional irreducible representations $E_g$ and $T_{2g}$. The outer product $E_g \otimes E_g$ decomposes into $A_{1g} \oplus A_{2g} \oplus E_g$, while $T_{2g} \otimes T_{2g}$ reduces to $ A_{1g} \oplus E_{g} \oplus T_{1g} \oplus T_{2g}$. The presence of the identity irreducible representation in both decompositions indicates the existence of two independent matrix elements corresponding to $\epsilon_e$ and $\epsilon_t$ the energies of V 3$d$ $E_g$ and $T_{2g}$ orbitals. Matrix elements coupling the different irreducible representations, $T_{2g} \otimes E_g \rightarrow T_{1g} \oplus T_{2g}$, do not contain the identity irreducible representation in the decomposition and, therefore, vanish. The Hamiltonian block describing zero-th neighbor overlap between V orbitals is thus:
\begin{equation}
H_0^{dd} =
\begin{bmatrix}
 \epsilon_{e} & 0 & 0 & 0 & 0 \\
 0 & \epsilon_{e} & 0 & 0 & 0 \\
 0 & 0 & \epsilon_{t} & 0 & 0 \\
 0 & 0 & 0 & \epsilon_{t} & 0 \\
 0 & 0 & 0 & 0 & \epsilon_{t} \\
\end{bmatrix}
\end{equation}

The second matrix in Eq. \ref{eq:ham_explicit} results from first-neighbor interactions. Consequentially, the Hamiltonian is subjected to the subgroup of symmetries  the bond $\bondvec$ invariant 

of the full $O_h$ point symmetry which leaves


\begin{equation}
H_1^{pd} = 
\begin{bmatrix}
-\sqrt{3} s_1 v_{ep} & 0            & \sqrt{3} s_3 v_{ep}  \\ % v5 = V(eg,p)
 s_1 v_{ep}          &-2 s_2 v_{ep} &  s_3 v_{ep}          \\ % v4 = V(t2g,p)
 s_2 v_{tp}          &  s_1 v_{tp}  & 0                    \\
0                    &  s_3 v_{tp}  &  s_2 v_{tp}          \\
 s_3 v_{tp}          & 0            &  s_1 v_{tp}          \\
\end{bmatrix}
\end{equation}

Second-neighbor interactions between the twelve equivalent species located at $\left[0\sfrac{1}{2}\sfrac{1}{2}\right]$. The stabilizer group is the Abelian group $C_{2v}$. The $O_h$ group has twelve $C_{2v}$ subgroups, of which six are maximal subgroups (the remaining six belong to the intermediate group $C_{4v}$). \cite{wadhawan_introduction_2000} The six maximal $C_2v$ subgroups stabilize the twelve second-neighbor bonds (two collinear bonds per subgroup). 



in which
\begin{widetext}
\begin{equation}
H_2^{dd} =
\begin{bmatrix}
        (c_{x} (c_{z}-2 c_{y})-2 c_{yz}) v_{6} /\sqrt{3}
                     +(c_{yz}+c_{x} (c_{y}+c_{z})) v_{7} &            c_{y} (c_{z}-c_{x}) v_{6}                         &                  -\sqrt{3} s_{xy} v_{8}  &     \sqrt{3} s_{yz} v_{8}                 &                                        0  \\
                               c_{y} (c_{z}-c_{x}) v_{6} &   (c_{yz}+c_{x} (c_{y}+c_{z})) v_{7}  -\sqrt{3} c_{zx} v_{6} &                           -s_{xy} v_{8}  &             -s_{yz} v_{8}                 &                           2 s_{zx} v_{8}  \\
                                  -\sqrt{3} s_{xy} v_{8} &                        -s_{xy} v_{8}                         & (c_{x}+c_{y}) c_{z} v_{11}+c_{xy} v_{9}  &             -s_{zx} v_{10}                &                             s_{yz} v_{10} \\
                                   \sqrt{3} s_{yz} v_{8} &                        -s_{yz} v_{8}                         &                           -s_{zx} v_{10} & c_{x} (c_{y}+c_{z}) v_{11} +c_{yz} v_{9}  &                            -s_{xy} v_{10} \\
                                                       0 &                       2 s_{zx} v_{8}                         &                           -s_{yz} v_{10} &             -s_{xy} v_{10}                & c_{y} (c_{x}+c_{z}) v_{11} +c_{zx} v_{9}  \\
\end{bmatrix}                                                                                                                                                                        
\end{equation}



\begin{equation}
H^{pp}_2 =
\begin{bmatrix}
               c_{yz} v_{12} +c_{x} (c_{y}+c_{z}) v_{14} &              -s_{xy} v_{13}                             &              -s_{zx} v_{13}                            \\
              -s_{xy} v_{13}                             &               c_{zx} v_{12} +c_{y} (c_{z}+c_{x}) v_{14} &              -s_{yz} v_{13}                            \\
              -s_{zx} v_{13}                             &              -s_{yz} v_{13}                             &               c_{xy} v_{12}+ c_{z} (c_{x}+c_{y}) v_{14} \\
\end{bmatrix}
\end{equation}
\end{widetext}

In Eqs. XX - XX,
\begin{align}
\begin{split}
c_x &= \cos(\pi a k_x) \\
c_y &= \cos(\pi a k_y) \\
c_z &= \cos(\pi a k_z)
\end{split}
\begin{split}
s_x &= \sin(\pi a k_x) \\
s_y &= \sin(\pi a k_y) \\
s_z &= \sin(\pi a k_z)
\end{split}
\end{align}
and
$s_{ij} = s_i s_j$ and $c_{ij} = c_i c_j$.




%\section{Film growth and characterization}

%
% WIGNER D MATRIX
%
\section{wigner D matrix} \label{appendix:wigner}




The matrix elements


Wigner matrices $D^{(l)}(R)$ corresponding to representations of the three-dimensional special orthogonal group SO(3) are obtained from:\cite{martin_electronic_2004}
\begin{equation}
\label{eq:imselfenergy}
% Shankar p 333, Exercise 12.5.7
% Wolfram p 85-87
% In the book, Euler angles are used. Here, I use Tait Bryan angles
D^{(l)}(R) = \bra{l',m'} R \ket{l,m}
\end{equation}
with rotations about axes $\hat{x}$, $\hat{y}$, and $\hat{z}$ parameterized by Tait Bryan angles $\theta$, $\phi$, and $\psi$:
\begin{equation}
R = e^{-i\theta\hat{L}_x/\hbar} e^{-i\phi\hat{L}_y/\hbar} e^{-i\psi\hat{L}_z/\hbar}
\end{equation}
%
In the $\hat{L}_y$ basis, the angular momenta matrix elements are:\cite{shankar_fundamentals_2014}
\begin{align}
\label{eq:angular_momenta}
% x y z <- z x y
% Shankar p 327-328, Eq. 12.5.21a-b
% Wolfram p 85-87
\hat{L}_x & = \frac{\hat{L}_{+}-\hat{L}_{-}}{2i} \hbar \\
\hat{L}_y & = \delta_{l,l'}\delta_{m,m'} \hbar m \\
\hat{L}_z & = \frac{\hat{L}_{+}+\hat{L}_{-}}{2} \hbar
\end{align}
with raising $\hat{L}_+$ and lowering $\hat{L}_-$ matrix elements:
\begin{equation}
\label{eq:raising_lowering_operator}
% Shankar p 327, Eq. 12.5.20
\hat{L}_{\pm} = \delta_{l,l'}\delta_{m\pm1,m'} \sqrt{(j\mp m)(j\pm m+1)} ,
\end{equation}

With the scalar $(-1)^l$ representing inversion, the full three-dimensional orthogonal group $O(3)$, which includes improper rotations, may be produced from the products:





\section{$O_h$}
The $O_h$(m$\bar{3}$m) point group consists of forty-eight symmetry operations which restore the NaCl-structure lattice onto itself. Twenty-four symmetries are proper rotations belonging to the chiral octahedral subgroup $O$(432). These are subdivided among five conjugacy classes\footnote{Classes $\class_i$ are mutually distinct sets containing symmetries $R_i$ and $R_j$ related by conjugation with $R_k$ an arbitrary group element: $R_kR_iR_k^{-1}=R_j$} --  $\id$, $3c_4^2$, $6c_2$, $8c_3$, and $6c_4$ -- which correspond to the identity, two-fold rotations about $\left<001\right>$, two-fold rotations around $\left<011\right>$, three-fold rotations about $\left<111\right>$, and four-fold rotations around $\left<001\right>$. The additional twenty-four symmetries are formed by compounding $O$ symmetries with the inversion operator $i$, are improper rotations\footnote{Improper rotations, or rotoinversions, are rotation operations followed by inversion.} and also exhibit a similar structure with five conjugacy classes: $i$, $3\sigma_2$, $6\sigma_2$, $8\sigma_6$, and $6\sigma_4$, associated with inversion $i$, reflection across $\{001\}$, reflection across $\{011\}$, three-fold roto-inversions around $\left<111\right>$, and four-fold roto-inversions about $\left<001\right>$. The group has two generators\footnote{The generating basis of a group is a minimal set of elements which is capable of producing all elements in the group when multiplied together.}, $c_4$ and $\sigma_6$, which fully determine all relations among matrix elements.

The vector $\bondvec = [00\frac{1}{2}]$ representing first neighbor bonds is stabilized by symmetries belonging to $C_{4v}$ (4mm). These include the identity $\id \seitz{1}{0}$ as well as a two-fold $c_{4}^2$ and two four-fold $c_{4}$ rotations around $\bondvec$: $\seitz{2_{001}}{0}$, $\seitz{2_{001}}{0}$, $\seitz{4^-_{001}}{0}$, and $\seitz{4^+_{001}}{0}$. In addition, there are four mirror planes, of which two are vertical-reflection planes $\sigma_v$, $\seitz{m_{010}}{0}$ and $\seitz{m_{100}}{0}$, 
and two are diagonal reflection planes $\sigma_d$, $\seitz{m_{\frac{1}{2}\bar{\frac{1}{2}}0}}{0}$ and $\seitz{m_{\frac{1}{2}\frac{1}{2}0}}{0}$. $\bondvec$ resides at the intersection of mirror plans. The generators of the group are $c_4$ and $\sigma_v$.

Second neighbor bonds $\bondvec = [0\frac{1}{2}\frac{1}{2}]$ are stabilized by the four symmetries of the $C_{2v}$ (2mm): the identity $\id \seitz{1}{0}$, the two-fold rotation $c_2 \seitz{ 2_{0\frac{1}{2}\frac{1}{2}} }{0}$, the vertical-mirror plane $\sigma_v \seitz{m_{001}}{0}$, and the diagonal mirror plane $\sigma_d \seitz{m_{0\frac{1}{2}\bar{\frac{1}{2}}}}{0}$. The latter two elements are group generators.

[should already have introduced representations and irreps]

$O_h$ exhibits ten irreducible representations $\Gamma^{(\mu)}$. These are labeled according to both Mulliken\footnote{In Mulliken notation, letters A,E,G, and H denote irreducible representations of dimensions one, two, three, and four. B is used in lieu of A for symmetries which are antisymmetric with respect to the principal axis (the axis of highest cyclical symmetry). Subscripts pairs $g$ and $u$, $1$ and $2$ as well as single and double primes are used to denote representations displaying even and odd parities with respect to inversion, horizontal two-fold rotations $c_2$, and horizontal reflections $\sigma_h$.} and BSW.


The group is non-Abelian with . The two group generators, $\sigma_6$ and $c_4$, fully define symmetry relations.
matrix element 


The three dimensional representations arise from non-commuting three-fold and six-fold symmetry elements. 




As required by translational symmetry, the crystallographic point groups exhibit integer characters.





$O_h$



All ten $O_h$ conjugacy classes produced by the direct product $O \otimes C_i$ are presented in the first row of Table \ref{table:chi}.


\section{Calculation of irreducible characters}

Characters of irreducible representations are determined from conjugacy class coefficients using the Burnside algorithm.\cite{burnside_theory_2010,mckay_construction_1970,unger_computing_2006,schneider_dixons_1990,dixon_high_1967} 

Characters corresponding to irreducible representation $\mu$ and conjugacy class i are thus obtained as:
\begin{equation}
\label{eq:irrep_characters}
\chi\left(\Gamma_i^\alpha\right) = \frac{d_\alpha}{n_i} \lambda_{i\alpha}
\end{equation}
in which number of elements in the class $i$ is $n_i$ and the dimensions of irreducible representation $\alpha$:
\begin{equation}
\label{eq:irrep_dimension}
d_\alpha = \sqrt{ \frac{g}{\sum_i \lambda_{i\alpha} \lambda_{i\alpha}^\dag / n_i }  }
\end{equation}
g is the order of the group.


The number of times a class $\class_k$ appears in the decomposition of products of classes $\class_i$ and $\class_j$ is $h_{jk}^{(i)}$. 

Class products decompose into sum of classes:
\begin{equation}
\label{eq:class_coefficients}
\class_i \class_j = \sum_k h_{jk}^{(i)} \class_k
\end{equation}



% simultaneous dec
The similarity transform $\Lambda$:
\begin{equation}
\Lambda^{-1} P \Lambda = \lambda
\end{equation}
which simultaneously diagonalizes the $i$ commuting expansion matrices $h_{jk}$ is determined using the matrix pencil:
\begin{equation}
\label{eq:matrix_pencil}
P_{jk} = \sum_i r_i h_{ijk}
\end{equation}
with arbitrary factors $r_i$ added to lift eigendegeneracies.




%
%
%
\section{Calculation of irreducible characters}

\begin{table}
\caption{\label{table:chi} $O_h$ representation characters}
\begin{ruledtabular}
\begin{tabular*}{10cm}{llrrrrrrrrrr}
\multicolumn{2}{c}{$O_h$}         &$\id$&3$c_4^2$& 6$c_2$ & 8$c_3$ & 6$c_4$ &  $i$ & 3$\sigma_2$ & 6$\sigma_2$ & 8$\sigma_6$ & 6$\sigma_4$ \\  
$A_{1g}$        & $\Gamma_{1}  $  &  1  &     1  &     1  &     1  &     1  &   1  &          1  &          1  &          1  &          1  \\         %  irrep =  1 
$A_{2g}$        & $\Gamma_{2}  $  &  1  &     1  &    -1  &     1  &    -1  &   1  &          1  &         -1  &          1  &         -1  \\         %  irrep =  2 
$E_g   $        & $\Gamma_{12} $  &  2  &     2  &     0  &    -1  &     0  &   2  &          2  &          0  &         -1  &          0  \\         %  irrep =  3 
$T_{1g}$        & $\Gamma_{15} $  &  3  &    -1  &    -1  &     0  &     1  &   3  &         -1  &         -1  &          0  &          1  \\         %  irrep =  5 
$T_{2g}$        & $\Gamma_{25} $  &  3  &    -1  &     1  &     0  &    -1  &   3  &         -1  &          1  &          0  &         -1  \\         %  irrep =  4 
$A_{1u}$        & $\Gamma_{1} '$  &  1  &     1  &     1  &     1  &     1  &  -1  &         -1  &         -1  &         -1  &         -1  \\         %  irrep =  7 
$A_{2u}$        & $\Gamma_{2} '$  &  1  &     1  &    -1  &     1  &    -1  &  -1  &         -1  &          1  &         -1  &          1  \\         %  irrep =  6 
$E_u   $        & $\Gamma_{12}'$  &  2  &     2  &     0  &    -1  &     0  &  -2  &         -2  &          0  &          1  &          0  \\         %  irrep =  8 
$T_{1u}$        & $\Gamma_{15}'$  &  3  &    -1  &    -1  &     0  &     1  &  -3  &          1  &          1  &          0  &         -1  \\         %  irrep =  9 
$T_{2u}$        & $\Gamma_{25}'$  &  3  &    -1  &     1  &     0  &    -1  &  -3  &          1  &         -1  &          0  &          1  \\ \hline  %  irrep = 10 
\multicolumn{2}{c}{$C_{4v}$}      &$\id$& $c_4^2$&        &        & 2$c_4$ &      & 2$\sigma_v$ &  $\sigma_d$ &             &             \\ 
$A_1$           & $\Delta_{1}  $  &  1  &     1  &     .  &     .  &     1  &   .  &          1  &          1  &          .  &          .  \\         %  irrep =  1
$A_2$           & $\Delta_{1}' $  &  1  &     1  &     .  &     .  &     1  &   .  &         -1  &         -1  &          .  &          .  \\         %  irrep =  4
$B_1$           & $\Delta_{2}  $  &  1  &     1  &     .  &     .  &    -1  &   .  &          1  &         -1  &          .  &          .  \\         %  irrep =  3
$B_2$           & $\Delta_{2}' $  &  1  &     1  &     .  &     .  &    -1  &   .  &         -1  &          1  &          .  &          .  \\         %  irrep =  2
$E$             & $\Delta_{5}  $  &  2  &    -2  &     .  &     .  &     0  &   .  &          0  &          0  &          .  &          .  \\ \hline  %  irrep =  5
\multicolumn{2}{c}{$C_{2v}$}      &$\id$&        &  $c_2$ &        &        &      &  $\sigma_v$ & $\sigma_d$  &             &             \\
$A_{1}$         & $\Sigma_{1}  $  &  1  &     .  &     1  &     .  &     .  &   .  &          1  &          1  &          .  &          .  \\
$A_{2}$         & $\Sigma_{2}  $  &  1  &     .  &     1  &     .  &     .  &   .  &         -1  &         -1  &          .  &          .  \\
$B_{1}$         & $\Sigma_{3}  $  &  1  &     .  &    -1  &     .  &     .  &   .  &         -1  &          1  &          .  &          .  \\
$B_{2}$         & $\Sigma_{4}  $  &  1  &     .  &    -1  &     .  &     .  &   .  &          1  &         -1  &          .  &          .  \\ \hline
\multicolumn{2}{c}{$C_{3v}$}      &$\id$&        &        & 2$c_3$ &        &      &             & 3$\sigma_v$ &             &             \\
$A_{1}$         & $\Lambda_{1} $  &  1  &     .  &     .  &     1  &     .  &   .  &          .  &          1  &          .  &          .  \\
$A_{2}$         & $\Lambda_{2} $  &  1  &     .  &     .  &     1  &     .  &   .  &          .  &         -1  &          .  &          .  \\
$E$             & $\Lambda_{3} $  &  2  &     .  &     .  &    -1  &     .  &   .  &          .  &          0  &          .  &          .  \\ \hline
\multicolumn{2}{c}{$p$          } &  3  &     -1 &    -1  &     0  &     1  &  -3  &          1  &          1  &          0  &         -1  \\
\multicolumn{2}{c}{$d$          } &  5  &      1 &     1  &    -1  &    -1  &   5  &          1  &          1  &         -1  &         -1  \\
\multicolumn{2}{c}{$p \oplus  d$} &  8  &      0 &     0  &    -1  &     0  &   2  &          2  &          2  &         -1  &         -2  \\
\multicolumn{2}{c}{$p \otimes p$} &  9  &      1 &     1  &     0  &     1  &   9  &          1  &          1  &          0  &          1  \\
\multicolumn{2}{c}{$p \otimes d$} & 15  &     -1 &    -1  &     0  &    -1  & -15  &          1  &          1  &          0  &          1  \\
\multicolumn{2}{c}{$d \otimes d$} & 25  &      1 &     1  &     1  &     1  &  25  &          1  &          1  &          1  &          1  \\
\end{tabular*}
\end{ruledtabular}
\end{table}


\begin{table}
\caption{\label{table:subduction} $O_h$ subductions}
\begin{ruledtabular}
\begin{tabular*}{10cm}{llll}
$O_h$    & $C_{4v}$         & $C_{2v}$                              &   $C_{3v}$            \\ \hline
$A_{1g}$ & $A_1$            & $A_{1}$                               &   $A_1$               \\ 
$A_{2g}$ & $B_1$            & $B_{1}$                               &   $A_2$               \\ 
$E_g   $ & $A_1 \oplus B_1$ & $A_{1} \oplus B_{1}$                  &   $E$                 \\ 
$T_{1g}$ & $A_2 \oplus E$   & $A_{2} \oplus B_{1} \oplus B_{2}$     &   $A_2 \oplus E$      \\ 
$T_{2g}$ & $B_2 \oplus E$   & $A_{1} \oplus A_{2} \oplus B_{2}$     &   $A_1 \oplus E$      \\ 
$A_{1u}$ & $A_2$            & $A_{2}$                               &   $A_2$               \\ 
$A_{2u}$ & $B_2$            & $B_{2}$                               &   $A_1$               \\ 
$E_u   $ & $A_2 \oplus B_2$ & $A_{2} \oplus B_{2}$                  &   $E$                 \\ 
$T_{1u}$ & $A_1 \oplus E$   & $A_{1} \oplus B_{1} \oplus B_{2}$     &   $A_1 \oplus E$      \\ 
$T_{2u}$ & $B_1 \oplus E$   & $A_{1} \oplus A_{2} \oplus B_{1}$     &   $A_2 \oplus E$      \\ 
\end{tabular*}
\end{ruledtabular}
\end{table}

Reducible representations are decomposed into $\alpha$ irreducible representations:
\begin{equation}
\label{eq:irrep_decomposition}
\Gamma = \bigoplus_\alpha n_\alpha \Gamma_{i\alpha}
\end{equation}

in which the expansion coefficients are:
\begin{equation}
\label{eq:irrep_decomposition_coefficients}
% Wolfram p 20
n_\alpha = \frac{1}{g} \sum_i h_i \chi\left(\Gamma_{i\alpha}\right)^\dag \chi\left(\Gamma_i\right)
\end{equation}
The sum is over classes $i$, with number of elements $h_i$, and characters corresponding to the $\alpha$ irreducible representation $\chi(\Gamma_i^\alpha)$. $g$ is the group order.





%
%
%
\section{Calculation of Tait Bryan angles from rotational matrices}
Tait Bryan angles $\theta$, $\phi$, and $\gamma$, corresponding to rotations around $\hat{z}$, $\hat{y}$, and $\hat{x}$ axes, are obtained from the rotational matrix:
\begin{equation}
\label{eq:rotation}
R = R_x(\theta)R_y(\phi)R_z(\gamma)
\end{equation}
via the relationships:
\begin{align}
\label{eq:tait_bryan_angles}
% This solution avoids gimbal lock cases and does not require conditional statements.
%i = 1; j = 2; k = 3
%euler(i) = atan2(R(l,k),R(k,k))
%euler(l) = atan2(-R(i,k),sqrt(R(i,i)**2+R(i,j)**2))
%s=sin(euler(1)); c=cos(euler(1))
%euler(k) = atan2(s*R(k,i)-c*R(l,i),c*R(l,j)-s*R(k,j))
\theta & = {\rm atan2}\left(R_{23},R_{33}\right) \\
\phi   & =-{\rm atan2}\left(-R_{13},\sqrt{R_{11}^2+R_{12}^2}\right) \\
\gamma & =-{\rm atan2}\left(R_{31}\sin{\theta}-R_{21}\cos{\theta},R_{22}\cos{\theta}-R_{32}\sin{\theta}\right)
\end{align}
in which ${\rm atan2}$ is the four-quadrant arctangent function and $R_{ij}$ are the elements corresponding to row i and column j of R. 


\end{document}

