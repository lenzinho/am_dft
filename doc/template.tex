\documentclass[preprint,showpacs,preprintnumbers,superscriptaddress,prb,floatfix,aps]{revtex4-1}
% \documentclass[twocolumn,showpacs,preprintnumbers,superscriptaddress,prb,floatfix,aps,10pt]{revtex4-1}

\usepackage{amsmath,amssymb}
\usepackage{graphicx,textcomp}
\usepackage{physics} % bra-ket
\usepackage{epstopdf}
\usepackage{subfigure}
\usepackage{xfrac} % nice frac

\usepackage[version=3]{mhchem}
\usepackage{siunitx}
\sisetup{
  separate-uncertainty,
  output-open-uncertainty=[
}

\usepackage{xcolor}
\definecolor{abm}{RGB}{190,220,170}
\usepackage{todonotes}
%
\newcommand{\abm}[1]{\textcolor{orange}{ \bf [Antonio: #1] }}
\renewcommand{\vec}[1]{\ensuremath{\mathbf{#1}}}


\begin{document}

\title{Anomalous temperature-dependent thermal conductivity in VN}

\author{A. B. Mei}
\affiliation{Department of Materials Science and the Materials Research Laboratory
University of Illinois, 104 South Goodwin, Urbana, IL 61801}

\author{A. Rockett}
\affiliation{Department of Materials Science and the Materials Research Laboratory
University of Illinois, 104 South Goodwin, Urbana, IL 61801}

\author{L. Hultman}
\affiliation{Department of Physics, Chemistry and Biology (IFM), Link\"oping University, SE-581 83, Link\"oping, Sweden.}

\author{I. Petrov}
\affiliation{Department of Materials Science and the Materials Research Laboratory
University of Illinois, 104 South Goodwin, Urbana, IL 61801}
\affiliation{Department of Physics, Chemistry and Biology (IFM), Link\"oping University, SE-581 83, Link\"oping, Sweden.}

\author{J. E. Greene}
\affiliation{Department of Materials Science and the Materials Research Laboratory
University of Illinois, 104 South Goodwin, Urbana, IL 61801}
\affiliation{Department of Physics, Chemistry and Biology (IFM), Link\"oping University, SE-581 83, Link\"oping, Sweden.}


%\email{olhel@ifm.liu.se}

\begin{abstract}
...
\end{abstract}

\maketitle







\section{Acknowledgements}

\bibliography{library}

\clearpage

\appendix

%\section{Film growth and characterization}

\section{Calculation of irreducible characters}

Characters of irreducible representations are determined from conjugacy class coefficients\cite{burnside_theory_2010,mckay_construction_1970,unger_computing_2006,schneider_dixons_1990,dixon_high_1967}.

Symmetry groups decompose into disjoint classes $C_i$ consisting of distinct and mutually conjugate symmetry elements. Two elements A and B belong to the same class if there exists a symmetry element $X$ in the group, for which $XA = BX$ is satisfied.

The product of two classes may be expanded over all classes
\begin{equation}
\label{eq:class_coefficients}
C_i C_j = \sum_k h_{ijk} C_k
\end{equation}
with expansion coefficients $h_{ijk}$.

% simultaneous dec
The similarity transform $\Lambda$:
\begin{equation}
\Lambda^{-1} P \Lambda = \lambda
\end{equation}
which simultaneously diagonalizes the $i$ commuting expansion matrices $h_{jk}$ is determined using the matrix pencil:
\begin{equation}
\label{eq:matrix_pencil}
P_{jk} = \sum_i r_i h_{ijk}
\end{equation}
with arbitrary factors $r_i$ added to lift eigendegeneracies.


Characters corresponding to irreducible representation $\alpha$ and conjugacy class i are thus obtained as:
\begin{equation}
\label{eq:irrep_characters}
\chi\left(\Gamma_i^\alpha\right) = \frac{d_\alpha}{n_i} \lambda_{i\alpha}
\end{equation}
in which number of elements in the class $i$ is $n_i$ and the dimensions of irreducible representation $\alpha$:
\begin{equation}
\label{eq:irrep_dimension}
d_\alpha = \sqrt{ \frac{g}{\sum_i \lambda_{i\alpha} \lambda_{i\alpha}^\dag / n_i }  }
\end{equation}
g is the order of the group.
 

\section{Calculation of irreducible representations}
Wigner matrices $D^{(l)}(R)$ corresponding to $(2l+1)\times(2l+1)$ irreducible matrix representations of the three-dimensional special orthogonal group SO(3) are obtained from:\cite{martin_electronic_2004}
\begin{equation}
\label{eq:imselfenergy}
% Shankar p 333, Exercise 12.5.7
% Wolfram p 85-87
% In the book, Euler angles are used. Here, I use Tait Bryan angles
D^{(l)}(R) = \bra{l',m'} R \ket{l,m}
\end{equation}
with rotations about axes $\hat{x}$, $\hat{y}$, and $\hat{z}$ parameterized by Tait Bryan angles $\theta$, $\phi$, and $\psi$:
\begin{equation}
R = e^{-i\theta\hat{L}_x/\hbar} e^{-i\phi\hat{L}_y/\hbar} e^{-i\psi\hat{L}_z/\hbar}
\end{equation}
%
In terms of raising $\hat{L}_+$ and lowering $\hat{L}_-$ operators:
\begin{equation}
\label{eq:raising_lowering_operator}
% Shankar p 327, Eq. 12.5.20
\hat{L}_{\pm}\ket{l,m} = \sqrt{(j\mp m)(j\pm m+1)}\ket{l,m\pm1},
\end{equation}
angular momenta matrix elements in the $\hat{L}_y$ basis:\cite{shankar_fundamentals_2014}
\begin{align}
\label{eq:angular_momenta}
% x y z <- z x y
% Shankar p 327-328, Eq. 12.5.21a-b
% Wolfram p 85-87
\hat{L}_x & =   \hbar \bra{l',m'} \frac{L_{+}-L_{-}}{2i} \ket{l,m} \\
\hat{L}_y & = m \hbar \braket{l',m'}{l,m} \\
\hat{L}_z & =   \hbar \bra{l',m'} \frac{L_{+}+L_{-}}{2} \ket{l,m}
\end{align}

Irreducible matrix representations corresponding to inversion are obtained as diagonal matrices with entries equal to $(-1)^l$. The product $(-1)^{l} D^{(l)}(R)$  can thus produce any irreducible representations of the full three-dimensional orthogonal group O(3).\cite{sharma_general_1979,el-batanouny_symmetry_2008} % Wooten p 604 bottom, under Eq. 14.164 



%
%
%
\section{Calculation of irreducible characters}
Reducible representations are decomposed into $\alpha$ irreducible representations:
\begin{equation}
\label{eq:irrep_decomposition}
\Gamma = \bigoplus_\alpha n_\alpha \Gamma_{i\alpha}
\end{equation}

in which the expansion coefficients are:
\begin{equation}
\label{eq:irrep_decomposition_coefficients}
% Wolfram p 20
n_\alpha = \frac{1}{g} \sum_i h_i \chi\left(\Gamma_{i\alpha}\right)^\dag \chi\left(\Gamma_i\right)
\end{equation}
The sum is over classes $i$, with number of elements $h_i$, and characters corresponding to the $\alpha$ irreducible representation $\chi(\Gamma_i^\alpha)$. $g$ is the group order.





\section{Calculation of Tait Bryan angles from rotational matrices}
Tait Bryan angles $\theta$, $\phi$, and $\gamma$, corresponding to rotations around $\hat{z}$, $\hat{y}$, and $\hat{x}$ axes, are obtained from the rotational matrix:
\begin{equation}
\label{eq:rotation}
R = R_x(\theta)R_y(\phi)R_z(\gamma)
\end{equation}
via the relationships:
\begin{align}
\label{eq:tait_bryan_angles}
% This solution avoids gimbal lock cases and does not require conditional statements.
%i = 1; j = 2; k = 3
%euler(i) = atan2(R(l,k),R(k,k))
%euler(l) = atan2(-R(i,k),sqrt(R(i,i)**2+R(i,j)**2))
%s=sin(euler(1)); c=cos(euler(1))
%euler(k) = atan2(s*R(k,i)-c*R(l,i),c*R(l,j)-s*R(k,j))
\theta & = {\rm atan2}\left(R_{23},R_{33}\right) \\
\phi   & =-{\rm atan2}\left(-R_{13},\sqrt{R_{11}^2+R_{12}^2}\right) \\
\gamma & =-{\rm atan2}\left(R_{31}\sin{\theta}-R_{21}\cos{\theta},R_{22}\cos{\theta}-R_{32}\sin{\theta}\right)
\end{align}
in which ${\rm atan2}$ is the four-quadrant arctangent function and $R_{ij}$ are the elements corresponding to row i and column j of R. 


\end{document}

